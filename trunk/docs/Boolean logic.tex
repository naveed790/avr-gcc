\documentclass[journal,12pt,twocolumn]{IEEEtran}
%
%\\usepackage{setspace}
\usepackage{textcomp}
\usepackage{gensymb}
\usepackage{xcolor}
\usepackage{caption}
%\usepackage{subcaption}
%\doublespacing
%\singlespacing

\usepackage{graphicx}
\graphicspath{{./images/}}
\usepackage[colorlinks=true, urlcolor=blue, linkcolor=black]{hyperref}
%\usepackage[parfill]{parskip}
%\usepackage{amssymb}
%\usepackage{relsize}
\usepackage[cmex10]{amsmath}
\usepackage{mathtools}
%\usepackage{amsthm}
%\interdisplaylinepenalty=2500
%\savesymbol{iint}
%\usepackage{txfonts}
%\restoresymbol{TXF}{iint}
%\usepackage{wasysym}
\usepackage{amsthm}
\usepackage{mathrsfs}
\usepackage{txfonts}
\usepackage{stfloats}
\usepackage{cite}
\usepackage{cases}
\usepackage{subfig}
%\usepackage{xtab}
\usepackage{hyperref}
\usepackage{longtable}
\usepackage{multirow}
%\usepackage{algorithm}
%\usepackage{algpseudocode}
\usepackage{enumitem}
\usepackage{mathtools}
%\usepackage{eenrc}
%\usepackage[framemethod=tikz]{mdframed}
%\usepackage{hyperref}
\usepackage{listings}
    \usepackage[latin1]{inputenc}                                 %%
    \usepackage{color}                                            %%
    \usepackage{array}                                            %%
    \usepackage{longtable}                                        %%
    \usepackage{calc}                                             %%
    \usepackage{multirow}                                         %%
    \usepackage{hhline}                                           %%
    \usepackage{ifthen}                                           %%
  %optionally (for landscape tables embedded in another document): %%
    \usepackage{lscape}     
\usepackage{tikz}
\usepackage{circuitikz}
\usepackage{karnaugh-map}
\usepackage{pgf}

\usepackage{url}
\def\UrlBreaks{\do\/\do-}



%\usepackage{stmaryrd}


%\usepackage{wasysym}
%\newcounter{MYtempeqncnt}
\DeclareMathOperator*{\Res}{Res}
%\renewcommand{\baselinestretch}{2}
\renewcommand\thesection{\arabic{section}}
\renewcommand\thesubsection{\thesection.\arabic{subsection}}
\renewcommand\thesubsubsection{\thesubsection.\arabic{subsubsection}}

\renewcommand\thesectiondis{\arabic{section}}
\renewcommand\thesubsectiondis{\thesectiondis.\arabic{subsection}}
\renewcommand\thesubsubsectiondis{\thesubsectiondis.\arabic{subsubsection}}



%\surroundwithmdframed[width=\columnwidth]{lstlisting}
\def\inputGnumericTable{}                                 %%
\lstset{
%language=C,
frame=single, 
breaklines=true,
columns=fullflexible
}
 

\begin{document}
%

\theoremstyle{definition}
\newtheorem{theorem}{Theorem}[section]
\newtheorem{problem}{Problem}
\newtheorem{proposition}{Proposition}[section]
\newtheorem{lemma}{Lemma}[section]
\newtheorem{corollary}[theorem]{Corollary}
\newtheorem{example}{Example}[section]
\newtheorem{definition}{Definition}[section]
%\newtheorem{algorithm}{Algorithm}[section]
%\newtheorem{cor}{Corollary}
\newcommand{\BEQA}{\begin{eqnarray}}
\newcommand{\EEQA}{\end{eqnarray}}
\newcommand{\define}{\stackrel{\triangle}{=}}
\vspace{2cm}
\title{ 
Logical Expression through avr-gcc
}

\author{Md. Naveed Ahmed}


\maketitle
\tableofcontents
\bigskip
%
%\newpage
\section{Problem Statement}



\textbf{Question-14}  : In the logic circuit shown in the figure, Y is given by
\begin{figure}[h]
    \centering
    \includegraphics[scale=0.8]{../figures/log_exp.png}\\
\end{figure}\\
\begin{enumerate}[label=(\alph*)]
    \item Y = ABCD
    \item Y = (A + B) (C + D)
    \item Y = A + B + C + D
    \item Y = AB + CD
\end{enumerate}

\section{\textbf{Components}}
\input{components}
\begin{table}[!h]
\centering
\caption{}
\label{table:7447_disp}
\end{table}
\section{\textbf{Solution :}}
\subsection{Theoretical Solution}
    Based on Demorgans Law
    \begin{center}
    \begin{equation}
    \overline{AB} = \overline{A}+\overline{B}
    \end{equation}
    \begin{equation}
    \overline{\overline{A}}  = A
    \end{equation}
    %\caption{Circuit}
    \end{center}
    As per the boolean circuit A,B,C and D are inputs and Y is the output. The equivalent expression of the boolean logic is \\
    \begin{align*}
    Y  = \overline{\overline{AB}.\overline{CD}}
    \end{align*}
    By using equation(1) then the output Y is
    \begin{align*}
    Y  = \overline{\overline{AB}}+\overline{\overline{CD}}
    \end{align*}
    Again by using equation(2) then the output Y is 
    \begin{align*}
    Y  = AB + CD
    \end{align*}

\subsection{Truth table for Boolean Logic}
\begin{table}[!h]
\begin{tabular}{|c|c|c|c|c|}
\hline
\centering
A & B & C & D & Y \\ 
\hline 
0 & 0 & 0 & 0 & 0\\
0 & 0 & 0 & 1 & 0\\
0 & 0 & 1 & 0 & 0\\
0 & 0 & 1 & 1 & 1\\
0 & 1 & 0 & 0 & 0\\
0 & 1 & 0 & 1 & 0\\
0 & 1 & 1 & 0 & 0\\
0 & 1 & 1 & 1 & 1\\
1 & 0 & 0 & 0 & 0\\
1 & 0 & 0 & 1 & 0\\
1 & 0 & 1 & 0 & 0\\
1 & 0 & 1 & 1 & 1\\
1 & 1 & 0 & 0 & 1\\
1 & 1 & 0 & 1 & 1\\
1 & 1 & 1 & 0 & 1\\
1 & 1 & 1 & 1 & 1\\
\hline
\end{tabular}
\centering 
\label{Truth table}
\caption{}

\end{table}

\pagebreak
\section{\textbf{Connections}}
\begin{table}[!h]
\begin{tabular}{|c|c|c|c|c|c|}
\hline
\centering
Input & A & B & C & D & \\ 
\hline 
Output &  &  &  &  & Y\\
\hline
Arduino & 2 & 3 & 4 & 5 & 13\\


\hline
\end{tabular}
\centering 
\label{}
\caption{}

\end{table}

\section{Procedure}

\subsection{LED Blinking}
\begin{enumerate}
\item Make connections as per TABLE-III
\item Connect Arduino ground to the led - resistor end
\item Connect Arduino 8 pin to the LED Positive
\item In arduino we are having pins A,B,C,D.here we are using port B pin 8 is taken as output pin.
\item port D pins 2,3,4,5 pins are taken as a inputs. portD pins 2,3,4,5 pins are connected vcc or gnd in breadboard as per truth table
\item Execute the following code
\item Observe the results as per below TABLE II by changing input values
\end{enumerate}


\textbf{Observe the circuit and verify the program by executing the link provided below.}\\
\begin{center}
\fbox{\parbox{8.5cm}{\url{https://github.com/naveed790/FWC/}}}
\end{center}
\end{document}
