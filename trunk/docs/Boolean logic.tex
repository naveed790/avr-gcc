\documentclass[journal,12pt,twocolumn]{IEEEtran}
%
%\\usepackage{setspace}
\usepackage{textcomp}
\usepackage{gensymb}
\usepackage{xcolor}
\usepackage{caption}
%\usepackage{subcaption}
%\doublespacing
%\singlespacing

\usepackage{graphicx}
\graphicspath{{./images/}}
\usepackage[colorlinks=true, urlcolor=blue, linkcolor=black]{hyperref}
%\usepackage[parfill]{parskip}
%\usepackage{amssymb}
%\usepackage{relsize}
\usepackage[cmex10]{amsmath}
\usepackage{mathtools}
%\usepackage{amsthm}
%\interdisplaylinepenalty=2500
%\savesymbol{iint}
%\usepackage{txfonts}
%\restoresymbol{TXF}{iint}
%\usepackage{wasysym}
\usepackage{amsthm}
\usepackage{mathrsfs}
\usepackage{txfonts}
\usepackage{stfloats}
\usepackage{cite}
\usepackage{cases}
\usepackage{subfig}
%\usepackage{xtab}
\usepackage{hyperref}
\usepackage{longtable}
\usepackage{multirow}
%\usepackage{algorithm}
%\usepackage{algpseudocode}
\usepackage{enumitem}
\usepackage{mathtools}
%\usepackage{eenrc}
%\usepackage[framemethod=tikz]{mdframed}
%\usepackage{hyperref}
\usepackage{listings}
    \usepackage[latin1]{inputenc}                                 %%
    \usepackage{color}                                            %%
    \usepackage{array}                                            %%
    \usepackage{longtable}                                        %%
    \usepackage{calc}                                             %%
    \usepackage{multirow}                                         %%
    \usepackage{hhline}                                           %%
    \usepackage{ifthen}                                           %%
  %optionally (for landscape tables embedded in another document): %%
    \usepackage{lscape}     
\usepackage{tikz}
\usepackage{circuitikz}
\usepackage{karnaugh-map}
\usepackage{pgf}

\usepackage{url}
\def\UrlBreaks{\do\/\do-}



%\usepackage{stmaryrd}


%\usepackage{wasysym}
%\newcounter{MYtempeqncnt}
\DeclareMathOperator*{\Res}{Res}
%\renewcommand{\baselinestretch}{2}
\renewcommand\thesection{\arabic{section}}
\renewcommand\thesubsection{\thesection.\arabic{subsection}}
\renewcommand\thesubsubsection{\thesubsection.\arabic{subsubsection}}

\renewcommand\thesectiondis{\arabic{section}}
\renewcommand\thesubsectiondis{\thesectiondis.\arabic{subsection}}
\renewcommand\thesubsubsectiondis{\thesubsectiondis.\arabic{subsubsection}}



%\surroundwithmdframed[width=\columnwidth]{lstlisting}
\def\inputGnumericTable{}                                 %%
\lstset{
%language=C,
frame=single, 
breaklines=true,
columns=fullflexible
}
 

\begin{document}
%

\theoremstyle{definition}
\newtheorem{theorem}{Theorem}[section]
\newtheorem{problem}{Problem}
\newtheorem{proposition}{Proposition}[section]
\newtheorem{lemma}{Lemma}[section]
\newtheorem{corollary}[theorem]{Corollary}
\newtheorem{example}{Example}[section]
\newtheorem{definition}{Definition}[section]
%\newtheorem{algorithm}{Algorithm}[section]
%\newtheorem{cor}{Corollary}
\newcommand{\BEQA}{\begin{eqnarray}}
\newcommand{\EEQA}{\end{eqnarray}}
\newcommand{\define}{\stackrel{\triangle}{=}}
\vspace{2cm}
\title{ 
Logical Expression through Avr-gcc
}

\author{Md. Naveed Ahmed}


\maketitle
\tableofcontents
\bigskip
%
%\newpage
\section{Abstract}

In the ciruit A,B,C and D are digital inputs, Y is digital output.The equivalent circuit shows the logical expression Y=AB+CD.
\begin{figure}[h]
    \centering
    \includegraphics[scale=0.8]{../figures/log_exp.png}\\
    \caption{Y=AB+CD}
\end{figure}\\
\section{\textbf{Components}}
\input{components}
\begin{table}[!h]
\centering
\caption{}
\label{table:7447_disp}
\end{table}
\section{\textbf{Solution :}}
\subsection{Theoretical Solution}
    Based on Demorgans Law
    \begin{center}
    \begin{equation}
    \overline{AB} = \overline{A}+\overline{B}
    \end{equation}
    \begin{equation}
    \overline{\overline{A}}  = A
    \end{equation}
    %\caption{Circuit}
    \end{center}
    As per the boolean circuit A,B,C and D are inputs and Y is the output. The equivalent expression of the boolean logic is \\
    \begin{align*}
    Y  = \overline{\overline{AB}.\overline{CD}}
    \end{align*}
    By using equation(1) then the output Y is
    \begin{align*}
    Y  = \overline{\overline{AB}}+\overline{\overline{CD}}
    \end{align*}
    Again by using equation(2) then the output Y is 
    \begin{align*}
    Y  = AB + CD
    \end{align*}
    \label{fig:circuit}
\begin{itemize}
    \item The figure given below is the pin diagram of Seven Segment Display
    \begin{figure}[h]
    \centering
    \includegraphics{../figures/seven.png}
    \caption{Seven segment display}
    \label{fig:my_label}
    \end{figure}
   \item  The table(Fig. 3) given abvoe is the connections between 7447 BCD Decoder and Seven Segment Display
   \begin{figure}
       \centering
       \includegraphics[scale=0.5]{../figures/sevenseg.png}
       \caption{}
       \label{fig:my_label}
   \end{figure}

    \item The diagram below shows the pin diagram of 7447 BCD Decoder.The output pins of 7447 is connected to Seven Segment Display using fig 3.
    \begin{figure}[h]
    \centering
    \includegraphics[scale=0.5]{../figures/ic7447.png}
    \caption{}
    \label{fig:my_label}
    \end{figure}
\end{itemize}
\subsection{Truth table for Boolean Logic}
\begin{table}[!h]
\begin{tabular}{|c|c|c|c|c|}
\hline
\centering
A & B & C & D & Y \\ 
\hline 
0 & 0 & 0 & 0 & 0\\
0 & 0 & 0 & 1 & 0\\
0 & 0 & 1 & 0 & 0\\
0 & 0 & 1 & 1 & 1\\
0 & 1 & 0 & 0 & 0\\
0 & 1 & 0 & 1 & 0\\
0 & 1 & 1 & 0 & 0\\
0 & 1 & 1 & 1 & 1\\
1 & 0 & 0 & 0 & 0\\
1 & 0 & 0 & 1 & 0\\
1 & 0 & 1 & 0 & 0\\
1 & 0 & 1 & 1 & 1\\
1 & 1 & 0 & 0 & 1\\
1 & 1 & 0 & 1 & 1\\
1 & 1 & 1 & 0 & 1\\
1 & 1 & 1 & 1 & 1\\
\hline
\end{tabular}
\centering 
\label{Truth table}
\caption{}

\end{table}


\section{Procedure}
\subsection{LED Blinking}
\begin{enumerate}
\item 1 Connect Arduino ground to the led - resistor end
\item Connect Arduino 13 pin to the LED Positive
\item Execute the following code
\item Observe the results as per below TABLE III by changing input values
\end{enumerate}
\subsection{Testing on Seven Segment Display using 7447}
\begin{enumerate}
\item connect the circuit using 7447 BCD-Seven segment display decoder and Arduino
\item  connect the seven segment pins to 7447 using fig 3.
\item  Make the connections according to TABLE II \\
\input{../docs/connections.tex}
\begin{table}[!h]
\centering
\caption{}
\label{table:7447_disp}
\end{table}
\item connect the output pins of 7447 to Ground(Gnd)
\item Verify the Boolean operation in Arduino using the following code and making
pin connections according to fig 2,3.
\end{enumerate}

\textbf{Observe the circuit and verify the program by executing the link provided below.}\\
\begin{center}
\fbox{\parbox{8.5cm}{\url{https://github.com/naveed790/FWC/}}}
\end{center}
\end{document}
